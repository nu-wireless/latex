\documentclass{article}

\usepackage{graphicx}   % for figures
\usepackage{hyperref}   % for hyperlinks

\title{Mini Report: Bunny RF Resistance Test}
\author{Your Name}
\date{November 3, 2025}

\begin{document}
\maketitle
\tableofcontents

\section{Introduction}
This brief report tests your ability to structure a document in \LaTeX{} using
sections, lists, equations, a figure, a table, and a URL.

\section{Methods}
\subsection{Materials and Steps}
\begin{itemize}
  \item Power supply
  \item Multimeter
  \item Connecting wires
\end{itemize}

\begin{enumerate}
  \item Apply a small voltage ($V$) across the bunny’s “antenna” section.
  \item Measure the resulting current ($I$) through the material.
  \item Compute resistance using Equation~(\ref{eq:ohm}).
\end{enumerate}

\section{Results}
According to Ohm’s Law, resistance $R$ is
\begin{equation}\label{eq:ohm}
R = \frac{V}{I}.
\end{equation}

\noindent Example measurements:
\begin{center}
\begin{tabular}{l l l}
Trial & $V$ (V) & $I$ (mA) \\
1 & 5.0 & 2.5 \\
2 & 5.0 & 2.4 \\
3 & 5.0 & 2.6 \\
\end{tabular}
\end{center}

\begin{figure}[h]
  \centering
  \includegraphics[width=0.5\textwidth]{bunny.jpg}
  \caption{Bunny during low-frequency resistance measurement.}
  \label{fig:bunny}
\end{figure}

As seen in Figure~\ref{fig:bunny}, the subject maintained a steady pose during testing.

\section{Conclusion}
This one-page report demonstrates basic \LaTeX{} skills: document structure, lists,
math, figures, tables, and hyperlinks. For more information, visit
\url{https://www.overleaf.com}.

\end{document}