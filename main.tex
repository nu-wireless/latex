% =============================================================================
% LaTeX Workshop Slide Deck, Northeastern University Wireless Club
% =============================================================================

% =============================================================================
% Beamer Theme Customizations
% =============================================================================
\documentclass[x11names]{beamer}                            % Global option [x11names] avoids xcolor clash
\usetheme{Madrid}
\usecolortheme{beaver}
\setbeamertemplate{navigation symbols}{}                    % Hides navigation buttons at bottom-right
\setbeamercovered{transparent}                              % Make hidden items (e.g., \onslide, \only) semi-transparent
\setbeamercolor{section in head/foot}{bg=gray!60, fg=black} % Set a darker gray-ish color for the "section in head/foot".
                                                            % This is the bottom-right footer column used for displaying 
                                                            % the version (Git commit hash) and the page numbering.


% =============================================================================
% Packages
% =============================================================================
\usepackage{graphicx}   % For including images (e.g., QR codes)
\usepackage{hyperref}   % For clickable hyperlinks
\usepackage{xcolor}     % For defining custom colors (e.g., the urlcolor)
\usepackage{iexec}      % Run command line (used for Git hash in versioning)
\usepackage{tcolorbox}  % For colored boxes
\usepackage{qrcode}     % For generating QR codes
\usepackage[utf8]{inputenc}
\usepackage{mdframed}   % For framed environments
\usepackage{minted}     % For syntax-highlighted code blocks
\usepackage{tikz}       % For drawing (used in mdframed titles)
\usepackage{lastpage}   % For referencing the total number of pages, used in footer as \pageref{LastPage}
                        % requires compiling the document twice to resolve the total page count correctly
% =============================================================================
% Versioning (Git commit hash for version info in footer)
% =============================================================================
\newcommand{\gitAbbrevHash}{\iexec{git rev-parse --short HEAD}}

% =============================================================================
% Custom Environments (Defined for future use, currently not used)
% =============================================================================
\newenvironment{question}[1][]{
  \ifstrempty{#1}{}{
    \mdfsetup{
      frametitle={
        \tikz[baseline=(current bounding box.east),outer sep=0pt]
        \node[anchor=east,rectangle,fill=gray!30]{#1};
      }}
  }
  \mdfsetup{
    innertopmargin=10pt,
    linecolor=gray!30,
    linewidth=2pt,
    topline=true,
    frametitleaboveskip=\dimexpr - \ht\strutbox\relax
  }
  \begin{mdframed}
}{\end{mdframed}}

% =============================================================================
% Color + Hyperlink Configuration
% =============================================================================
\definecolor{nuRed}{HTML}{C8102E}  % Northeastern primary red

\hypersetup{
  colorlinks=true,
  linkcolor=blue,
  urlcolor=nuRed
}

% =============================================================================
% Footer Configuration
% =============================================================================
\setbeamertemplate{footline}{%
  \leavevmode%
  \hbox{
    \begin{beamercolorbox}[wd=.35\paperwidth,ht=2.25ex,dp=1ex,leftskip=.3cm]{author in head/foot}%
      Northeastern University Wireless Club%
    \end{beamercolorbox}%
    \begin{beamercolorbox}[wd=.20\paperwidth,ht=2.25ex,dp=1ex,center]{title in head/foot}%
      \LaTeX\ Workshop%
    \end{beamercolorbox}%
    \begin{beamercolorbox}[wd=.15\paperwidth,ht=2.25ex,dp=1ex,center]{date in head/foot}%
      Fall 2025%
    \end{beamercolorbox}%
    \begin{beamercolorbox}[wd=.30\paperwidth,ht=2.25ex,dp=1ex,leftskip=.3cm, rightskip=.3cm plus1fil]{section in head/foot}%
      Version: \gitAbbrevHash{} \hspace{1em} \insertpagenumber/\pageref{LastPage}%
    \end{beamercolorbox}}%
  \vskip0pt%
}

% =============================================================================
% Compilation Notes
% =============================================================================
% Since the footer now uses \insertpagenumber/\pageref{LastPage} (instead of
% \insertframenumber), LaTeX must compile the document twice to correctly resolve
% the total page count reference provided by the 'lastpage' package.
% (First pass writes the page labels; second pass reads and resolves them.)

% =============================================================================
% Title Page Information 
% =============================================================================
\title{\LaTeX\ Workshop}
\subtitle{Fall 2025}
\author{
  Northeastern University Wireless Club - W1KBN \\
  {\small Muhammad Elarbi, \textit{[Name Here]}, \textit{[Name Here]}}
}
\date{November 3, 2025}

\begin{document}

% =============================================================================
% Slide: Title Page + Sign-In with QR Code
% =============================================================================
\begin{frame}
  \titlepage
  \vspace{-1cm}
  \begin{center}
    \textbf{Sign in Here:} \\ 
    \qrcode{https://forms.gle/EukxnYUPSvckBFsg7} \\
    {\small \href{https://nuwireless.org/signin}{nuwireless.org/signin}}
  \end{center}
\end{frame}

% =============================================================================
% Slide: About
% =============================================================================
\begin{frame}
  \onslide<1->{
    \begin{tcolorbox}[colframe=black, colback=blue!10, title=About, center title]
    \begin{itemize}
      \item Regular Meetings: Thursdays, 7 PM @ Hayden Hall, Room 503
      \item Workshops: Mondays, 7:10 PM @ Forsyth Building, Room 236
    \end{itemize}        
  \end{tcolorbox}
  }

  \onslide<2->{
    \begin{tcolorbox}[colframe=black, colback=blue!10, title=Upcoming Workshops, center title]
    \begin{center}
      \renewcommand{\arraystretch}{1.2} % spacing between rows
      \begin{tabular}{|l|c|c|}
        \hline
        \textbf{Workshop} & \textbf{Day} & \textbf{Date} \\ \hline
        Git & Monday & November 10, 2025 \\ \hline
        Tesla Coils (w/ IEEE NU) & Monday & November 17, 2025 \\ \hline
      \end{tabular}
    \end{center}

    \vspace{0.5em}
    \centering
    \textbf{Both workshops:} 7:10 PM @ Forsyth Building, Room 236
  \end{tcolorbox}
  }
\end{frame}

% =============================================================================
% Slide: Workshop Goals and Structure
% =============================================================================
\begin{frame}{Workshop Goals and Structure}
    \textbf{Goals:}
    \begin{itemize}
        \item<1-> Understand what \LaTeX{} is and why it matters for students in STEM.
        \item<2-> Learn how to write and compile \LaTeX{} documents using Overleaf and other editors.
        \item<3-> Walk away having created your first \LaTeX\ document or knowing how to improve your existing ones.
    \end{itemize}
    \textbf{Structure:}
    \begin{enumerate}
        \item<4-> What is \LaTeX{}? \textit{(The What)}
        \item<5-> Why use \LaTeX{}? \textit{(The Why)}
        \item<6-> Writing and Compiling \LaTeX{} \textit{(The How)}
        \begin{itemize}
          \item<6-> Hands-On Examples in Overleaf
        \end{itemize}
        \item<7-> Helpful Resources \& Tips
    \end{enumerate}
\end{frame}

% =============================================================================
% Slide: What is LaTeX
% =============================================================================
\begin{frame}{What is \LaTeX{}}
\begin{itemize}
    \item A typesetting system.
    \item Based on plain text and compiled into PDFs.
    \item Especially powerful for complex formatting, math, citations, figures, and \textit{code}.
    \item Common in academia, research, and publishing - many professors encourage it!
\end{itemize}
\begin{tcolorbox}[colback=gray!10, colframe=gray!40, title=From learnlatex.org]
\LaTeX{} can be scary for new users as it is not a word processor, and because it is not a single program.
\end{tcolorbox}
% \LaTeX{} can be scary for new users as it is not a word processor, and because it is not a single program.\footnote{\url{https://www.learnlatex.org/en/}}
\end{frame}


% =============================================================================
% Slide: TeX vs LaTeX
% =============================================================================
\begin{frame}{\TeX\ vs \LaTeX{}}
  \textbf{\href{https://tug.org/}{\TeX}}: A typesetting engine created by Donald Knuth in the late 1970s.
  \begin{itemize}
    \item Low-level, extremely powerful, but hard to use directly.
    \item Similar to how Git is the underlying version control tool.
  \end{itemize}
  
  \textbf{\href{https://www.latex-project.org/}{\LaTeX}}: A markup language and set of macros built on top of \TeX\ (by Leslie Lamport).
  \begin{itemize}
    \item You write LaTeX code and compile it using a TeX system (like Overleaf, TeXShop, etc.)
    \item Makes \TeX\ accessible, structured, and easier to use for most people.
    \item Similar to how GitHub provides a user-friendly ``interface" to Git.
  \end{itemize}
\end{frame}

% =============================================================================
% Slide: Why Use LaTeX over something like Microsoft Word?
% =============================================================================
\begin{frame}{Why Use \LaTeX{} over something like Microsoft Word?}
    \begin{columns}[t] % <-- align the tops of the columns, so both columns start at the same vertical height
        \begin{column}{0.48\textwidth}
        Microsoft Word is \textbf{WYSIWYG}\\[0.1em] \scriptsize{\textbf{W}hat \textbf{Y}ou \textbf{S}ee \textbf{I}s \textbf{W}hat \textbf{Y}ou \textbf{G}et}
        \begin{itemize}
            \item Click-and-drag formatting
            \item Can be inconsistent
            \item Formatting breaks easily
            \item Great for casual documents
        \end{itemize}
        \end{column}
        \begin{column}{0.48\textwidth}
        \LaTeX{} is \textbf{WYSIWYT}\\[0.1em] \scriptsize{\textbf{W}hat \textbf{Y}ou \textbf{S}ee \textbf{I}s \textbf{W}hat \textbf{Y}ou \textbf{T}ype}
        \begin{itemize}
            \item Precise, consistent output
            \item Separates content from formatting
            \item Portable, version-controlled
            \item Professional, especially for STEM
        \end{itemize}
        \end{column}
    \end{columns}
    \end{frame}    

% =============================================================================
% Slide: Real-World Applications of LaTeX
% =============================================================================
\begin{frame}{Real-World Applications of \LaTeX}
\begin{itemize}
    \item Resumes and CVs
    \item Lab reports and technical documentation
    \item Research papers, journal articles
    \item Thesis/dissertation formatting
    \item Slide decks, like this one (\LaTeX\ Beamer)!
\end{itemize}
\end{frame}


% =============================================================================
% Slide: How to Write and Compile LaTeX
% =============================================================================
\begin{frame}{How to Write and Compile \LaTeX}
  % Option 1: Online
  \only<1>{
    \begin{tcolorbox}[colback=gray!10, colframe=black, title=Online (Simplest)]
      \textbf{Overleaf}: runs entirely in your browser, no setup, collaborative, and free Pro with Northeastern University email!
    \end{tcolorbox}
    \begin{center}
      \Large We'll be focusing on Overleaf today.
    \end{center}
  }
  % Option 2: Local Development
  \only<2>{
    \begin{tcolorbox}[colback=gray!10, colframe=black, title=Local Development (Advanced)]
      \scriptsize
      \textbf{Requires two parts:}
      \begin{enumerate}
        \item \textbf{A \LaTeX{} Distribution}: provides compilers like \texttt{pdflatex} and all core/built-in packages like \texttt{amsmath}.
        \begin{itemize}
          \item macOS: \texttt{MacTeX}, Linux: \texttt{TeX Live}
          \item Windows: \texttt{MiKTeX} or \texttt{TeX Live}
        \end{itemize}
        \item \textbf{An Editor or IDE}: where you write your \LaTeX\ code.
        \begin{itemize}
          \item VS Code + \href{https://marketplace.visualstudio.com/items?itemName=James-Yu.latex-workshop}{\texttt{LaTeX Workshop}} (general-purpose code editor with \LaTeX\ extension)
          \item TeXstudio (cross-platform LaTeX IDE)
          \item TeXShop (macOS LaTeX IDE included with MacTeX)
          \item Vim + \texttt{vimtex} / Emacs + \texttt{AUCTeX} (advanced text editors with LaTeX plugins)
        \end{itemize}
      \end{enumerate}
      The editor calls a compiler (e.g., \texttt{pdflatex}) provided by the distribution to generate your PDF.
    \end{tcolorbox}
    }
\end{frame}

% =============================================================================
% Slide: Why Overleaf is Perfect for Beginners
% =============================================================================
\begin{frame}{Why Overleaf is Perfect for Beginners}
    \begin{itemize}
        \item Handles all the heavy lifting “under the hood”; Overleaf takes care of compiling, packages, and setup so you can just focus on writing.
        \item Cloud-based, no install required.
        \item Real-time preview and auto-compile.
        \item Supports collaboration; great for group projects.
        \item Version control built-in.
        \item Connect your Northeastern email for \textbf{free Overleaf Pro}. \href{https://www.overleaf.com/edu/northeastern}{overleaf.com/edu/northeastern}
    \end{itemize}
\end{frame}    

% =============================================================================
% Slide: Basic Document Structure
% =============================================================================
\begin{frame}[fragile]{Basic Document Structure}
    \begin{columns}[t] % <-- align the tops of the columns, so both columns start at the same vertical height
        \begin{column}{0.55\textwidth}
          \textbf{Source (\LaTeX\ Code):}
          % Keep the code inside minted flush-left (no indentation) so it renders fully left-aligned
          % on the slide for optimal readability when compiled.
          % Minted is verbatim, so any leading spaces will appear in the output.
          \begin{minted}[fontsize=\scriptsize]{latex}
\documentclass{article}
\begin{document}
Hello, world!
\end{document}
          \end{minted}
        \end{column}
        \begin{column}{0.45\textwidth}
          \textbf{Output:}
          \vspace{1em}
          \fbox{\parbox{.9\linewidth}{Hello, world!}}
        \end{column}
      \end{columns}
\end{frame}

    
% =============================================================================
% Slide: Common LaTeX Environments
% =============================================================================
\begin{frame}[fragile]{Common LaTeX Environments}
    \begin{columns}[t] % <-- align the tops of the columns, so both columns start at the same vertical height
        \begin{column}{0.55\textwidth}
        \textbf{Source (\LaTeX\ Code):}
        % Keep the code inside minted flush-left (no indentation) so it renders fully left-aligned
        % on the slide for optimal readability when compiled.
        % Minted is verbatim, so any leading spaces will appear in the output.
        \begin{minted}[fontsize=\scriptsize]{latex}
\begin{itemize}
    \item Bullet points
\end{itemize}

\begin{enumerate}
    \item Ordered list
\end{enumerate}

\begin{equation}
    E = mc^2
\end{equation}
        \end{minted}
        \end{column}

        \begin{column}{0.45\textwidth}
        \textbf{Output:}
        \vspace{0.5em}
        \fbox{
          \parbox{.9\linewidth}{
            \begin{itemize}
              \item Bullet points
            \end{itemize}
            \begin{enumerate}
              \item Ordered list
            \end{enumerate}
            \vspace{1em}
            \begin{equation}
                E=mc^2
            \end{equation}
          }
        }
        \end{column}
    \end{columns}
    \vspace{1cm}
    ... and more here: \href{https://www.overleaf.com/learn/latex/Environments}{overleaf.com/learn/latex/Environments}
\end{frame}

\begin{frame} {\LaTeX\ Quirks}
  \begin{itemize}
    \item Quotes
    \item Delimeters
  \end{itemize}
\end{frame}

% =============================================================================
% Slide: Live Example / Demo
% =============================================================================
\begin{frame}{Let's Head to Overleaf!}
\centering
\Huge{\textbf{Time for a live demo!}}\\[1em]
\Large{\href{https://www.overleaf.com/}{overleaf.com}}
\end{frame}

% =============================================================================
% Slide: Resources and Tools
% =============================================================================
\begin{frame}{Resources and Tools}
    \begin{itemize}
        \item \textbf{Documentation}: \href{https://www.overleaf.com/learn}{overleaf.com/learn}, excellent for learning both Overleaf and \LaTeX\ fundamentals, even if you develop locally.
        \item \textbf{Tutorials}: \href{https://tug.org/interest.html#doc}{Extensive list} by TUG (\TeX\ Users Group), Youtube tutorials by \href{https://www.youtube.com/playlist?list=PLHXZ9OQGMqxcWWkx2DMnQmj5os2X5ZR73}{Dr. Trefor Bazett}
        \item \textbf{Templates}: \href{https://github.com/melarb/latex-template}{Muhammad's} homework template, \href{https://www.overleaf.com/latex/templates}{overleaf.com/latex/templates} or \href{https://latextemplates.com/}{latextemplates.com}
        \item \textbf{Cheat Sheets}: \href{https://quickref.me/latex.html}{quickref.me/latex.html} or \href{https://wch.github.io/latexsheet/}{wch.github.io/latexsheet}
        \item \textbf{Detexify}: \href{https://detexify.kirelabs.org/classify.html}{detexify.kirelabs.org}, draw a symbol to get its corresponding \LaTeX\ command.
        \item \textbf{CTAN (Comprehensive \TeX\ Archive Network)}: \href{https://ctan.org/}{ctan.org}, the main repository for all \LaTeX\ packages, especially useful for local development.
        \item Essential packages for math, units, syntax highlighting, and circuit diagrams: \texttt{amsmath}, \texttt{siunitx}, \texttt{minted}, \texttt{circuitikz}
    \end{itemize}
\end{frame}

% =============================================================================
% Slide: Final Thoughts
% =============================================================================
\begin{frame}{Final Thoughts}
    \begin{itemize}
        \item \LaTeX\ may seem intimidating at first, but it's a powerful tool for creating high-quality, consistent documents.
        \item Overleaf makes it accessible to everyone, but as your skills grow, we encourage you to explore developing locally to better understand what's happening under the hood.
        \item Embrace the learning curve, your future self will thank you!
    \end{itemize}
    \vspace{1em}
    This entire slide deck was compiled locally using \LaTeX\ (Beamer)!\\[1em]
    \textbf{This workshop is revived after its last run in Fall 2020. Special thanks to \href{https://jack.leightcap.com/}{Jack Leightcap} (President '22) and Connor Northway (VP '22).}
\end{frame}

% =============================================================================
% Slide: Contact Information
% =============================================================================
\begin{frame}
    \frametitle{Contact Us}
    \begin{itemize}
        \item Questions? Feel free to reach out!
        \item Workshop Team Emails: \\
        {\href{mailto:elarbi.m@northeastern.edu}{elarbi.m}, 
        \href{mailto:last-name.x@northeastern.edu}{last-name.x}, 
        \href{mailto:last-name.y@northeastern.edu}{last-name.y}}[at]northeastern[d0t]edu
        \item General Workshop Email: \href{mailto:workshops@nuwireless.org}{workshops}[at]nuwireless[d0t]org
        \item Website: \href{https://nuwireless.org}{nuwireless.org}
        \item Location: Hayden Hall, Room 503
    \end{itemize}
    \vspace{1cm}
    \begin{flushright}
        \footnotesize{© 2025 Northeastern University Wireless Club} \\
        \footnotesize{Design: \href{https://melarbi.com}{Muhammad Elarbi}, based on \LaTeX\ Beamer}\\
    \end{flushright}
\end{frame}

\end{document}
